%!TEX program = xelatex
\documentclass[a4paper,UTF8]{article}
\usepackage{geometry}
\usepackage{ctex}
\geometry{left=2.0cm,right=2.0cm,top=1.4cm,bottom=1.4cm}
\title{一带一路与债务陷阱}
\author{胡扬,冻思易,安南}
\begin{document}
	\maketitle
	\begin{abstract}
		这里是摘要
	\end{abstract}
	\section{债务陷阱论的由来}
		债务陷阱论,又名债务陷阱外交(Debt-trap Diplomacy),该词汇由印度地缘政治专家 Brahma Chellaney 创造。在 Brahma Chellaney 的定义中,债务陷阱外交是指利用债务手段进行的外交,往往用来达到一些其他的目的。在这种外交手段中,债权国故意向债务国提供过多的贷款,其目的是在债务国无法偿还债务时迫使其在经济或政治上作出让步。
		\subsection{IMF与西方国家的债务陷阱}
			国际货币基金组织(IMF)成立于1944年,是面临国际收支困难的国家 的最后贷款人,或者说是濒临破产国家的最后一根救命稻草。在1980年代,一些第三世界国家因无力偿还1970年代从西方商业银行获得的巨额贷款转而求助于IMF。IMF往往会慷慨的为这些国家提供贷款,但随之而来的也是相当严苛的条件。一般来讲IMF会要求各国政府进行严厉的经济调整,例如削减社会服务支出,终止对食品和燃料等基本物品的价格补贴,推广经济自由化、私有化。然而,这种经济调整往往是灾难性的。在埃及,人民正承受着IMF经济改革计划的冲击。IMF要求政府削减补贴和首次引入增值税后,基本食品、水和燃料的价格飙升。在这种冲击之下,该国出售食物残渣的现象越来越普遍,人们开始抢购有缺陷的食品。在巴基斯坦,来自IMF的贷款导致该国掀起一股自由化、私有化和放松管制的浪潮。自上世纪80年代以来,巴基斯坦已经出售了160多家国有实体。这导致了数十万人失业、生活水平降低和公共开支的大幅削减。这两个国家经历了这么多痛苦之后,我们反而看到了外债和负债的显著增加。来自IMF和西方国家的贷款并没有让第三世界国家摆脱债务,反而使其陷入了债务陷阱。
		\subsection{一带一路与债务陷阱}
			虽然债务陷阱这个名字所指的行动已被许多西方国家和国际货币基金组织多次使用,但目前它最常与中国和它的一带一路倡议联系在一起。这种联系始于2017年1月,印度地缘政治专家 Brahma Chellaney 发表了一篇题为《中国的债务陷阱外交》的文章。该文章认为中国的一带一路项目的意图在于利用债务推进战略利益,通过给予相关政府无法偿还的巨额贷款,使它们跳入了债务陷阱,迫使它们出卖本国利益给中国。同年7月,为了缓解债务压力,斯里兰卡与中国签署关于汉班托塔港的管理开发协议,中国招商局控股港口有限公司(CMPort)向斯里兰卡政府支付11.2亿美元购得汉班托塔港口70\%的股权,并将租用该港口以及周边1.5万英亩土地建设工业园区,租期为99年。此事件在传到西方国家和印度后迅速发酵,从此中国的一带一路倡议便与债务陷阱的标签联系在了一起。
	\section{一带一路是否会导致债务陷阱}
		认为一带一路倡议会导致受助国跳入债务陷阱的评论家一般会利用部分一带一路沿线国家债务水平上升的现象来支持这种论断,比如斯里兰卡、巴基斯坦等国阶段出现的债务问题。然而,有文献指出,这种论调并不可靠。来自司尔亚司数据信息有限公司全球数据库及世界银行数据库数据表明,一带一路沿线国家平均政府债务率在 2015年之后不仅没有显著提高,反而呈现总体下降趋势,且始终保持在50\% 以下的水平.从年均涨幅看,2014-2017年沿线国家债务存量占比平均增长 6.16\%,小于非沿线国家的6.33\%。可见,一带一路倡议提出以来,沿线国家平均债务水平并未出现明显的上升态势。由此可见,这些评论家犯了以片面看全面、夸大事实的错误。另外还有一点需要说明,即国有实体不等于政府。例如在斯里兰卡汉班托塔港事件中,与斯里兰卡政府进行交易的实体是CMPort而非中国政府。在这个交易中,CMPort获得使用权,斯里兰卡政府获得资金,这与一个西方投资者来中国投资建厂的行为没有任何本质区别。通过将CMPort与中国政府等价来对中国政府和它提出的一带一路倡议进行攻击实际上犯了“稻草人”的逻辑谬误。同时也有不少文献指出,一带一路倡议通过改善基础设施环境与扩大出口规模,非但没有使沿线国家落入债务陷阱,反而对其具有一定降债效应。
	\section{造成“债务陷阱论”误解的重要原因}
		“债务陷阱论”如今在西方媒体界以及舆论界广为流传,甚至在一带一路沿线国家的政界民间引起了误解。该论调的产生以及流传甚广的其中一个重要原因,在于我国在推进一带一路倡议的过程中,未能有效地与沿线国家做好沟通交流工作,这些问题主要表现为以下三个方面。
		
		\subsection{国外各界对于“一带一路倡议”初衷的误读}
			自2013年中国提出建设“新丝绸之路经济带”和“21世纪海上丝绸之路”的合作倡议起,国外各界就对这一倡议的动机进行了各种猜测,其中不乏许多负面的猜测、并由此产生了各种误读。该倡议提出的初期,外媒就对于“战略”一词过分解读,无中生有地指出其中的“侵略”意味。而随着我国一带一路沿线国家合作项目的推进,国外文献中涉及中国的对外援助和投资的看法更多趋于负面,有观点认为中国的一带一路倡议服务于地缘政治的目的。维斯娜·富尔(Veasna Var)和苏文达·波(Sovinda Po)认为,中国向斯里兰卡等国提供的软实力贷款令人联想到中国古时的朝贡体系。
		
			以上观点的产生,都是基于西方的思维模式对中方所作所为的解释,而忽略了作为倡议发起国的中国的思维方式。西方文化多注重个体主义,强调个体的利益与自由。因此,当西方各界尝试理解一带一路倡议,首先会考虑该倡议对于中国是否有裨益。其次,西方的分析思维更符合二元对立论,即“零和博弈”的思考方式。由于西方国家大多通过战争和掠夺起家,“一方受益、一方受损”的行为模式是他们故有的他们的思考方式。因此,当西方国家假设中国从一带一路倡议中获益,他们可能会得出“该倡议将会损害与中国合作的各个国家的利益”的结论,而“债务陷阱论”刚好贴合了这一结论。然而,中国的传统文化则更注重集体主义。以综合思维方式为出发点,中国在国内以及国际上秉持“合作共赢”的理念。因此中国强调整体性,强调“人类命运共同体”,尝试走和平发展的道路,并且希望能与各国一道促进经济交流发展;而这些观点很难让西方思维模式下的各国感同身受,因此 “以己度人”造成了误解。
		
		\subsection{中西媒体宣传效力的差距较大}
			自2017年印度学者布拉马·切拉尼(Brahma Cellani)提出“债务陷阱论”起,西方媒体对这一论题密切关注并广泛报道。该论题同时也引起了许多西方学者的关注,并且在西方学界乃至国外各学界引起讨论。然而,我国媒体和学界对于其中一些污名化内容的辩驳总显得力不从心,以至于流言甚嚣尘上、在国外舆论界反而占了主导地位。中国的国外宣传工作的弱势主要体现在以下几个方面:
			第一,国际舆论界中,中国媒体的故有影响力远小于西方媒体;西方媒体在国际舆论界中仍占据主导地位。中国的官方媒体虽然一直在为一带一路倡议的相关内容进行澄清,然而官媒在国外的宣传一直受到诸多限制,宣传效力大大削减。一方面,经由中文报道翻译而成的英文报道容易产生歧义,从而造成误解;例如,倡议提出伊始,西方学界对于“strategy”一词背后是否含有某种战略目的的质疑。另一方面,在与市场经济相关的论述方面,西方国家作为各类经济学理论的产生地,在话语权上有着天生的。国外各界对于中国的社会主义市场经济仍有不少怀疑之声。因此,在经济援助和经济合作产生的问题方面,中国缺少自身的理论体系说服外界。
			
			第二,中方在应对西方媒体的污名化论述时显得捉襟见肘。西方媒体善于抓住报道对象的“痛点”并片面化强调地进行报道,从而一定程度上进行舆论导向。在斯里兰卡汉班托塔港的例子中,西方媒体认为斯里兰卡因无力偿还中国的贷款而被迫将港口主权拱手相让于中国。该论调抓住了较为敏感的主权问题,并且片面强调了斯里兰卡与中国之间的债务关系,却刻意忽视了斯里兰卡的外债中,中国只占了10.6\%、低于日本在斯的债务的事实。该类报道模式在西方媒体中并不少见,即针对事实的一个方面进行曲解分析,从而得出实际上不符合逻辑的结论。而中国的学者和各类媒体虽然对该事实进行澄清,但主要宣传有效范围局限在国内。出于一些原因,国外各界尤其是民间对于中国官媒在一定程度上并不信任。然而,中国学界与部分非官方媒体相较于官媒受到的政治标签影响较小,且具有一定的对外发表观点的渠道。因此,中国学界与部分非官方媒体本应作为倾听国际声音、发出中国声音的窗口。但是当下中国学界习惯于注重于使用中文、在国内进行讨论,而在利用英文或其他外语、在国外学术圈中发表观点方面留下了相当大的空白。而非官方媒体则在许多时候不敢、不愿意在这些敏感问题上发出评论。同时,无论是中国学界还是非官方媒体均有偏爱从中方角度叙事,或依赖理论分析、案例研究与对西方质疑的被动批判反驳,却鲜有从合作国民众或舆情的角度进行深入分析,主动出击的情况。这些现状致使外国各界很难直接了解中国学者的观点,从而产生了误解,也导致了中方在舆论上始终处于相当被动的状态。因而在关于一带一路的对于宣传和误解澄清工作中,中国依旧有很多工作要做。
			
			第三,中国的对外宣传工作经验不足,对于一带一路沿线国家的民众宣传并不到位。中国一带一路倡议的提出和施行主要由国家主导,在与各沿线国家对接时也主要为高层、官方层次的联系。因而,中国与一带一路各国的民间联系并不是很紧密,在“五通”中最核心的“民心相通”上的工作不足。在一带一路倡议的推进过程中,中国较为注重与各国的政府外交,而关于公共外交方面的工作并不是很到位。在2016年的一项对于一带一路沿线国家民众的采访中,许多民众对于中国和一带一路倡议都所知甚少,部分民众甚至从未听说过一带一路倡议。忽视与沿线国家的民间交流、未充分考虑当地情况及民众意愿,导致了一带一路倡议实行伊始的一些问题和矛盾,该问题将在后文详述。
			
			综上,尽管中国一直对于一带一路倡议进行客观宣传,在国外却收效甚微。近年来虽然以上情况得到了一定程度的改善,但也仍有可以改进之处。
		
		\subsection{中国与一带一路沿线国家的合作方式的误区}
			一带一路倡议是多边合作的勇敢尝试,在各国合作的磨合过程中,由于合作方式产生的误区,难免会出现一些摩擦。然而,这些问题却成为了外媒支持“债务陷阱论”的论据。
			首先,中国一带一路倡议的参与主体多为国企,而国企投资易引起国外对于一带一路倡议目的的猜疑。目前中国企业在一带一路沿线国家的投资方向主要在于基建等大规模投资领域,中方与沿线国家的私企尚且不具备相关条件,而国企在这一方面毋庸置疑有着诸多优势。然而,国企虽是作为经济个体,在国外人士眼中却带有国家的政治标签,易引起负面的战略意图猜测。以中斯合作建造的汉班托塔港为例,中斯政府在签署协议后由中国招商局控股港口有限公司购得70\%的股份并租借99年。该经济协定引发了西方对其政治意图的猜测,2018年美国国务卿蒂勒森表示“中国利用掠夺性贷款使相关国家陷入债务泥潭,从而削弱其主权。”尽管中国国企与合作国家是通过正当途径协商后达成了经济协定,却依旧无法摆脱其中引起的政治意味的质疑。
			
			其次,中国在一带一路倡议初期未充分考虑合作国家当地的情况,合作项目未能达到预期,从而引发相关质疑。中国在斯的项目很多采用工程总承包(EPC)模式。中斯合作之初,两国在诸多标准方面存在分歧,而中方企业确实曾从国内直接带去大量原料和设备,许多项目岗位也由中国工人担任,这一点引起了斯方不满。EPC模式也不利于项目细节的本地化调整,导致经济发展水平、市场规模、运营能力不匹配等问题,因而未能达到预期效果,也由此引起项目实用性方面的质疑。同时,中企“一步到位”的“承包”方式也易引起关于两国合作的平等性的质疑。该类质疑在经过媒体炒作后极易上升到政治层面,从而成为“债务陷阱论”的论据。
		
	\section{中国如何消除“债务陷阱论”等误解}
		虽然一带一路倡议如今面临较为负面的国际舆论形势,中国仍可以从多个方面促进与各合作国之间的沟通交流,从而于症结所在之处遏制诸如“债务陷阱论”的负面流言。
		
		第一,中方在与一带一路沿线国家投资时,应当更多地考虑到当地国家的历史与文化,避免在该国的敏感问题上产生不必要的歧义。例如对于绝大部分一带一路沿线国家而言,其大都是曾经长期处于西方列强的殖民统治之下的小国。介于西方列强在近200年间在埃及与巴拿马均出现过通过债务问题对对方国家主权与领土完整造成巨大侵害的事件,而来自IMF对巴基斯坦、埃及与玻利维亚等国的贷款也对这些国家的主权造成了一定的侵害。因此在与中国进行“实力不对称”的合作时,其大都处于既欢迎来自中国的大量投资,又对自己国家的主权问题有所担忧。在如此情况下,中方在与对方合作的时候需要刻意注意减少相关涉及敏感问题的表述,或刻意与西方在这方面的的斑斑劣迹有所区别,尽可能避免在这些敏感问题上引起歧义,从根源上尽可能减少他国能够抓住中方投资时的“小辫子”不断炒作问题恶意污名化中国的可能性。
		
		第二,在构建“一带一路”国际话语权时,需要更加强调非官方媒体与中国学界在世界的影响力。现有的官媒主导的对外宣传模式,虽然能较为准确地传达国内的声音,却在国外各界缺乏公信力。而中国学界与非官方媒体又缺少对外通过“自述”与“述他”的方式主动对外宣传的意识与魄力,国情又致使中国通过网络上的大量自媒体进行对外宣传较为困难。因此中国学者与非官方媒体应当被鼓励在世界主流英文学术界与媒体界勇敢地发声,并在不断重申中方观点的同时,重视从合作国家的角度进行叙事,在对质疑或潜在的质疑点做出解释的同时,也使得当地的民众与世界各国人民能够通过更加生动的方式理解中国的和平共处五项原则与对外经济技术援助八项原则,更好地理解中国的外交政策。
		
		第三,注重公共外交,加强与沿线国家的民间交流。一带一路倡议中的“五通”包括:政策沟通、设施联通、贸易畅通、资金融通和民心相通。目前中国与沿线国家在前四点已经取得了一定成效,然而民间交流依旧不足,导致一带一路沿线国家的一些普通民众对一带一路项目产生了误解。比较极端的例子有:误将一带一路的合作项目理解为无偿援助,或认为中国的相关项目有负面的战略目的等。而中国的民众对于沿线国家也了解甚少,在中方民营企业与一带一路合作国家的合作方面导致了不少困难、使得难以进一步扩大交流层面。因此,一带一路倡议的推进过程中,公共外交不可或缺。
		
		第四,中国与沿线国家合作时,应充分考虑到当地的实际情况,促进平等的合作。EPC模式对于中企相对而言是最高效的模式,然而却未充分考虑当地的资源条件和经济现状,最终造成了“好心办坏事”,甚至在经过西方媒体炒作后上升到政治层面的污名化。因此,中方企业在与沿线国家制定相关合作协议之始,就应充分考虑当地的资源情况、市场规模、经济状况等,从而在现有条件的基础之上最大程度上实现合作共赢。
		
		
	\section{参考文献}
\end{document}